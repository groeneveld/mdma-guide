\newglossaryentry{schema}{
  name={Schema},
  description={A single learned model about the world or self that generates, in response to specific triggers, behavioral impulses, emotional responses, thoughts, and sometimes sensory perception~\cite{eckerUnlocking}.}
  %defined_in={sec:trauma},
  %stem={schema}
}

\newglossaryentry{reconsolidation-exhaustion}{
  name={Reconsolidation Exhaustion},
  description={Large amounts of reconsolidation cause intense mental and physical exhaustion that can feel different from regular exhaustion. It's often called a therapy hangover or EMDR hangover (in eye movement desensitization and reprocessing therapy). The phenomenon hasn't been formally studied. Anecdotal reports suggest it dissipates within a few hours to a couple of days.}
  %defined_in={session},
  %stem={emotionally exhausted, reconsolidation exhaustion, therapy hangover, emotional exhaustion}
}

\newglossaryentry{reconsolidation}{
  name={Reconsolidation},
  description={When a schema/implicit-memory first forms, it is \textit{consolidated}~\cite{eckerUnlocking}. When contradictory evidence causes sufficient mismatch or prediction error, the schema enters a state where it can be changed (it becomes \textit{labile}). Maintaining that mismatch or prediction error gradually updates the schema to reflect the new information. At the end of this process the schema \textit{re-consolidates,} and becomes unchangeable again. We use the term \textit{reconsolidate} to refer to this whole process of activation, updating, and reconsolidation.}
  %defined_in={sec:mechanism},
  %stem={reconsolidat}
}

\newglossaryentry{defense-cascade}{
  name={Defense Cascade},
  description={A series of physiological changes in the sympathetic and parasympathetic nervous systems that prepares the body to respond to imminent threats~\cite{kozlowskaDefenseCascade}. It includes arousal, flight-or-fight, freezing, and tonic/collapsed immobility. Increasing levels of perceived threat, threat imminence, powerlessness, and somatic sensory input activate these responses, though the order of activation also depends on individual variability and past experience.}
  %defined_in={sec:defensecascade},
  %stem={defense cascade}
}

\newglossaryentry{arousal}{
  name={Arousal},
  description={Increased alertness and threat assessment activated when a potential threat is noticed~\cite{kozlowskaDefenseCascade}. It is also preparation for flight-or-fight if the threat escalates. Heart rate, breathing, and muscle tone are increased. Saliva is no longer produced, and core muscles tighten to stabilize posture.}
  %defined_in={sec:defensecascade},
  %stem={arous}
}

\newglossaryentry{flight-or-fight}{
  name={Flight-or-fight},
  description={The active defense response characterized by high levels of adrenaline and muscle activation, highly increased heart and respiratory rates, and decreased pain sensitivity~\cite{kozlowskaDefenseCascade}.}
  %defined_in={sec:defensecascade},
  %stem={flight-or-fight}
}

\newglossaryentry{freeze}{
  name={Freeze},
  description={A flight-or-fight response temporarily put on hold~\cite{kozlowskaDefenseCascade}. Attention to the environment remains high but the body is frozen to avoid the notice of predators who are more likely to notice moving prey.}
  %defined_in={sec:defensecascade},
  %stem={freez}
}

\newglossaryentry{immobility}{
  name={Tonic/Collapsed Immobility},
  description={The inactive defense responses characterized by detachment, emotional and physical numbing, and immobility~\cite{kozlowskaDefenseCascade}. Predators are more attracted to moving prey and may lose interest in seemingly dead bodies. Collapsed immobility may escalate to unconsciousness.}
  %defined_in={sec:defensecascade},
  %stem={immobil}
}

\newglossaryentry{dissociation}{
  name={Dissociation},
  description={In this book we strictly define dissociation as chemical numbing (usually endogenous opioids) to align with the defense cascade literature and to mechanistically differentiate chemical numbing from automatic, unnoticed \glsdisp{avoidance}{avoidance}, which could also fit the ICD definition\footnote{\enquote{Involuntary disruption or discontinuity in the normal integration of one or more of the following: identity, sensations, perceptions, affects, thoughts, memories, control over bodily movements, or behaviour}~\cite{icd11}.} of dissociation.}
  %defined_in={sec:defensecascade},
  %stem={dissociat}
}

\newglossaryentry{trauma}{
  name={Trauma},
  description={Events that interact with existing schemas and resources to produce schemas, whether newly formed or reinforced, that may be adaptive in the short term but impair functioning or emotional health in the long term. Put another way, trauma is distressing events or chronic conditions that overwhelm our ability to cope, where our ability to cope depends on our capabilities and resources.}
  %defined_in={sec:trauma},
  %stem={trauma}
}

\newglossaryentry{connectogen}{
  name={Connectogen},
  description={A term for MDMA's class of drug, designed to unify the competing terms \textit{entactogen} and \textit{empathogen}~\cite{stockerConnectogen}. MDMA facilitates profound connection to self, body, senses, and others. The ego and sense of self are maintained and hallucinations are minimal, unlike with the classic psychedelics or hallucinogens.}
  %defined_in={sec:introduction},
  %stem={Connectogen}
}

\newglossaryentry{psychedelic}{
  name={Psychedelic},
  description={\enquote{Showing the mind/soul,} from the Greek \enquote{psyche} and \enquote{deloun}~\cite{osmondObit}. Most are tryptamines, phenethylamines, and ergamides~\cite{gumpper2024chemistry}. Different classes have different effects. We specifically do not categorize MDMA as a psychedelic, but rather a \glsdisp{connectogen}{connectogen}, though it is commonly grouped with other psychedelics.}
  %defined_in={na},
  %stem={psychedelic}
}

\newglossaryentry{attachment-theory}{
  name={Attachment Theory/Styles},
  description={Attachment theory describes how secure attachment in early childhood is critical to healthy emotional and psychological development~\cite{brownAttachmentDisturbances}. Severe failures in parenting cause non-secure attachment, which are complex networks of persistent maladaptive schemas. These failures are usually unintentional, done without full awareness of the consequences, and caused by lack of parental emotional capacity.}
  %defined_in={attachment},
  %stem={secure attachment, attachment theory, attachment disorder, disordered attachment, anxious attachment, insecure attachment}
}

\newglossaryentry{predictive-processing}{
  name={Predictive Processing},
  description={The prevailing model of brain function~\cite{clark2015surfing}. The brain internally models the world and self via complex networks of learned priors/beliefs to plan the future fulfillment of basic needs such as bodily integrity, reproduction, community, etc. Prediction error is a discrepancy between the brain's model of the world and incoming sensory data or between two contradictory models.\newline\newline Memory research has converged on the same phenomenon from a different angle with different terms~\cite{eckerUnlocking}. In that formulation, the brain learns models of the world and self called schemas. A mismatch is a contradiction between two schemas or between one schema and incoming sensory input.}
  %defined_in={sec:trauma},
  %stem={predicti, mismatch}
}

\newglossaryentry{resistance}{
  name={Resistance},
  description={Opposition to reconsolidation or other therapeutic processes that would actually be healthy for the individual.}
  %defined_in={na},
  %stem={resistance}
}

\newglossaryentry{bypass}{
  name={Spiritual Bypass},
  description={As \textcite{cashwell2007Bypass} states, spiritual bypass \enquote{occurs when a person attempts to heal psychological wounds at the spiritual level only and avoids the important (albeit often difficult and painful) work at the other levels, including the cognitive, physical, emotional, and interpersonal.} This results in persistence of maladaptive patterns and interrupted psychological development.}
  %defined_in={na},
  %stem={spiritual bypass}
}

\newglossaryentry{avoidance}{
  name={Avoidance},
  description={Physical or mental (often unnoticed and automatic) actions that direct attention away from contradictory information or distressing thoughts and inhibit reconsolidation.}
  %defined_in={sec:trauma},
  %stem={avoida}
}

\newglossaryentry{biopsychosocial-model}{
  name={Biopsychosocial Model},
  description={Prevailing model of mental illness as a complex interaction of biology (genes and medical history), psychology (schemas, in our view), and society (one's support system and social models of how to respond to adversity)~\cite{engel1977need}.}
  %defined_in={sec:trauma},
  %stem={biopsychosocial}
}

\newglossaryentry{complex-system}{
  name={Complex System},
  description={Systems with many interacting parts that are difficult to model and understand~\cite{hayes2020complex}. The overall behavior of the system can't be easily predicted by looking at the individual components. The mind/brain is a complex system.}
  %defined_in={sec:complex},
  %stem={complex system}
}

\newglossaryentry{attractor-state}{
  name={Attractor},
  description={A state in a complex system that the system is drawn to and which it is difficult to get out of~\cite{hayes2020complex}. Many mental illnesses are attractor states. Maladaptive schemas seem to stick around when they're paired with some reconsolidation-inhibiting process (like avoidance), which together form an attractor state~\cite{hayes2020complex,berghSelfEvidencing}. The durability of the state requires multiple reinforcing elements; removing those elements destabilizes it.}
  %defined_in={sec:complex},
  %stem={attractor}
}

\newglossaryentry{destabilization}{
  name={Destabilization},
  description={Two phenomena described in \ref{sec:complex}. The first is increased fluctuation between states of good and bad mental health that often precedes a stabilizing transition to the good state. The second is a transition to a stable and even worse state of mental health, possibly precipitated by confronting disturbing memories that were previously avoided. Both phenomena are part of the healthy process of therapy when managed well and entered into at an appropriate time.}
  %defined_in={sec:mechanism},
  %stem={destabiliz}
}

\newglossaryentry{afterglow}{
  name={Afterglow},
  description={A days-to-weeks long period after a session involving well-being, positive mood, mindfulness, positive behaviors, and less mental illness~\cite{evansAfterglow}. Conventional therapy might be more effective in this period. Anecdotal reports indicate that the occurrence of afterglow is unreliable. It can also occasionally be strong enough that people feel like they're on a low dose of MDMA.}
  %defined_in={afterglow},
  %stem={afterglow}
}