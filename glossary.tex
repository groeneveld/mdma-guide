\newglossaryentry{schema}{
  name={Schema},
  description={A single learned prediction (or prior) in the brain's world model \cite{eckerUnlocking}. What are called low-level (physically lower in the brain) schemas perform a vast array of functions relating to maintenance of basic bodily functions and sensory data processing \cite{clark2024experience}. Therapeutically-relevant schemas are generally more abstract, high-level predictions about the self, relationships, or whether the world is generally safe/predictable or not.}
  %defined_in={What is Trauma and Why do its Effects Stick Around?},
  %stem={schema}
}

\newglossaryentry{reconsolidation-exhaustion}{
  name={Reconsolidation Exhaustion},
  description={Emotional exhaustion and lack of energy follow successful reconsolidation (See Section \ref{def:hangover}). Often called therapy hangover. We think this is reliable enough to indicate that emotional exhaustion is a solid sign of therapeutic success. People seem to be capable of 1-2 hours of reconsolidation a day before exhaustion becomes so intense that the process is no longer possible.}
  %defined_in={The MDMA Therapy Session},
  %stem={emotionally exhausted, reconsolidation exhaustion, therapy hangover, emotional exhaustion}
}

\newglossaryentry{reconsolidation}{
  name={Reconsolidation},
  description={When a schema/memory is first formed it is "consolidated \cite{eckerUnlocking}." When prediction error on that schema becomes large enough, the schema enters a mode where it is changeable. Maintaining that prediction error then updates the schema to reflect the new information. At the end of this process the schema "re-consolidates" and becomes unchangeable again. We use "reconsolidate" to denote this whole process of activation, updating, and reconsolidation.}
  %defined_in={Mechanism of Healing},
  %stem={reconsolidat}
}

\newglossaryentry{defense-cascade}{
  name={Defense Cascade},
  description={A series of physiological changes that prepares the body to respond to immanent threats \cite{kozlowskaDefenseCascade}. Includes arousal, flight or flight, freezing, and immobility. Different situations and past experiences activate different responses.}
  %defined_in={Defense Cascade},
  %stem={defense cascade}
}

\newglossaryentry{arousal}{
  name={Arousal},
  description={The first step when a potential threat is noticed and assists in further assessing that threat \cite{kozlowskaDefenseCascade}. It is also preparation for more intense defense responses like fight or flight, freeze, or immobility. Heart rate, breathing, and muscle tone increase, saliva is no longer produced, and core muscles tighten to stabilize posture.}
  %defined_in={Defense Cascade},
  %stem={arous}
}

\newglossaryentry{fight-or-flight}{
  name={Fight or Flight},
  description={Active defense response characterized by high levels of adrenaline and muscle activation, increased heart rate, and decreased pain sensitivity \cite{kozlowskaDefenseCascade}.}
  %defined_in={Defense Cascade},
  %stem={fight-or-flight}
}

\newglossaryentry{freeze}{
  name={Freeze},
  description={A fight or flight response temporarily put on hold \cite{kozlowskaDefenseCascade}. One remains highly attentive but frozen to avoid the notice of predators who are more likely to notice moving objects.}
  %defined_in={Defense Cascade},
  %stem={freez}
}

\newglossaryentry{immobility}{
  name={Tonic/Collapsed Immobility},
  description={Inactive defense responses characterized by detachment, emotional and physical numbing, and immobility \cite{kozlowskaDefenseCascade}. Predators are more attracted to moving prey and may lose interest in seemingly-dead bodies. May escalate to unconsciousness.}
  %defined_in={Defense Cascade},
  %stem={immobility}
}

\newglossaryentry{dissociation}{
  name={Dissociation},
  description={Emotional numbing caused by brain-produced opioids in response to perceived threat and powerlessness (usually a maladaptive schema when you're not in an acutely dangerous situation) \cite{lanius2018review,kozlowskaDefenseCascade}. This can escalate to immobility and greater degrees of detachment from one's self and external reality.}
  %defined_in={Defense Cascade},
  %stem={dissociat}
}

\newglossaryentry{trauma}{
  name={Trauma},
  description={We use two closely related definitions: 1) Events that lead to over-generalized schemas that impair functioning or emotional health. 2) Distressing events or chronic conditions that overwhelm our ability to cope, where our ability to cope depends on our capabilities and resources \cite{laneReconsolidation}.}
  %defined_in={What is Trauma and Why do its Effects Stick Around?},
  %stem={trauma}
}

\newglossaryentry{predictive-processing}{
  name={Predictive Processing},
  description={The prevailing model of brain function \cite{clark2024experience}. The brain internally models the world (via complex layers of learned predictions) to better plan for the fulfillment of basic needs such as bodily integrity, reproduction, community, etc. Prediction error is a discrepancy between 1) the brain's model of the world and incoming sensory data, or 2) two contradictory model predictions \cite{clark2024experience}. Minimization of prediction error is the brain's core optimization function, achieved through the construction of more complex and accurate world-models.}
  %defined_in={What is Trauma and Why do its Effects Stick Around?},
  %stem={predicti}
}

\newglossaryentry{psychedelic}{
  name={Psychedelic},
  description={Showing the Mind/Soul, from the Greek 'psyche' and 'deloun' \cite{osmondObit}. Most are tryptamines, phenethylamines, and ergamides \cite{gumpper2024chemistry}. Among a variety of effects specific to each compound, they generally relax high-level predictions \cite{carhart2019rebus}. Psychedelic-Assisted Therapy uses this effect, and other effects specific to certain drugs (like MDMA's safety and empathy), to help change maladaptive predictions/schemas.}
  %defined_in={na},
  %stem={psychedelic}
}

\newglossaryentry{window-of-tolerance}{
  name={Window of Tolerance},
  description={The range of dissociation, arousal, or fight-or-flight where reconsolidation is possible. High levels of these states often inhibit reconsolidation \cite{razviPSIP}. MDMA expands the window beyond what is usable in regular psychotherapy.}
  %defined_in={na},
  %stem={window of tolerance}
}

\newglossaryentry{therapeutic-destabilization}{
  name={Therapeutic Destabilization},
  description={A period of instability experienced while transitioning from a stable state of mental illness to a stable state of mental health \cite{olthofDestabilization}. This should not be used as an excuse for unethical or unskilled behavior from therapists or guides.}
  %defined_in={Mechanism of Healing},
  %stem={destabiliz}
}

\newglossaryentry{attachment-theory}{
  name={Attachment Theory/Styles},
  description={Attachment theory is a model which posits that emotionally secure attachments formed in the first 18 months of life serve as the foundation for emotional and psychological development throughout one's life \cite{brownAttachmentDisturbances}.}
  %defined_in={Attachment Theory},
  %stem={secure attachment, attachment theory, attachment disorder, disordered attachment}
}

\newglossaryentry{mismatch}{
  name={Mismatch},
  description={The conscious contradiction of an active schema via either sensory input or another schema \cite{eckerUnlocking}.}
  %defined_in={Mechanism of Healing},
  %stem={mismatch}
}

\newglossaryentry{resistance}{
  name={Resistance},
  description={Opposition to reconsolidation or a broader therapeutic process that would actually be healthy for the individual. This is difficult to ascertain, as many therapeutic processes are not actually a good match for many people.}
  %defined_in={na},
  %stem={resistance}
}

\newglossaryentry{contraindication}{
  name={Contraindication},
  description={Any medical condition, life circumstance, activity, or medication that makes MDMA use particularly risky.}
  %defined_in={na},
  %stem={contraindicat}
}

\newglossaryentry{non-dual-awareness}{
  name={Non-dual Awareness},
  description={Experiences of unity, without the usual separation into self and other \cite{metzingerElephant}. We suspect that MDMA can produce states of partial non-dual awareness.}
  %defined_in={na},
  %stem={non-dual}
}

\newglossaryentry{ait}{
  name={Adverse Idealizing Transference (AIT)},
  description={Idealizing Transference is a phenomena in which clients develop strong positive feelings towards their therapist \cite{hook2018boundary,transferranceLoveHarm}. Sometimes this idealization can be intense enough that an unscrupulous or unskilled therapist may exploit (intentionally or not) the client for emotional, sexual, or financial gain, creating severe trauma for the client.}
  %defined_in={How to Find a Skilled and Well-Matched Therapist or Guide},
  %stem={adverse idealizing transference}
}

\newglossaryentry{avoidance}{
  name={Avoidance},
  description={Physical or mental activity that directs attention away from distressing schemas and inhibits reconsolidation. Short term avoidance can be healthy if used to temporarily postpone dealing with a problem until you have more capacity.}
  %defined_in={What is Trauma and Why do its Effects Stick Around?},
  %stem={avoida}
}

\newglossaryentry{psychogenic-illness}{
  name={Psychogenic Illness},
  description={Illnesses caused in large part by maladaptive schemas \cite{harris2018deepest}.}
  %defined_in={Physiological Health Effects of Trauma},
  %stem={psychogenic}
}

\newglossaryentry{selective-inhibition}{
  name={Selective Inhibition},
  description={The suppression of all voluntary distractions, avoidance, and coping strategies to highlight maladaptive schemas \cite{razviPSIP}. Razvi proposes that this also facilitates a "completion" of the defense cascade cycle, but we are uncertain about this.}
  %defined_in={Handling Dissociation and Avoidance During the Session},
  %stem={selective inhibition}
}

\newglossaryentry{grounding-techniques}{
  name={Grounding Techniques},
  description={Activities that turn off or turn down defense cascade activations. These usually involve distraction (e.g. name all the round objects in the room), feelings of safety (e.g. vividly recalling memories of safety), or feelings of power (e.g. vividly constructing mental imagery of overcoming some adversity) \cite{grounding}. As far as we can tell, they are not based in rigorous evidence, but people like them, they're easy, and they seem low-risk.}
  %defined_in={Managing Anxiety, Dissociation, and Other Adverse Symptoms Outside the Session},
  %stem={grounding}
}

\newglossaryentry{alexithymia}{
  name={Alexithymia},
  description={Consistent difficulty in noticing, identifying, and describing emotions \cite{hogeveen2021alexithymia}. This inhibits reconsolidation. The causes are not well-established.}
  %defined_in={Efficacy of MDMA-Therapy},
  %stem={alexithymia}
}

\newglossaryentry{biopsychosocial-model}{
  name={Biopsychosocial Model},
  description={Prevailing model of mental illness as a complex interaction of biology (genes and medical history), psychology (schemas, in our view), and society (one's support system and social models of how to respond to adversity) \cite{engel1977need}.}
  %defined_in={What is Trauma and Why do its Effects Stick Around?},
  %stem={biopsychosocial}
}

\newglossaryentry{spiritual-bypass}{
  name={Spiritual Bypass},
  description={The use of spiritual attainments, practices, or beliefs as reasons to not notice, investigate, or address one's maladaptive schemas \cite{cashwell2007Bypass}.}
  %defined_in={Making Sense of the Experience},
  %stem={spiritual bypass}
}