\newglossaryentry{schema}{
  name={Schema},
  description={A single learned prediction (also called priors or beliefs) in the brain's world model that is combined with an emotional reaction and possibly an episodic memory \cite{eckerUnlocking}. What are called low-level (physically lower in the brain) schemas perform a vast array of functions relating to the maintenance of basic bodily functions and sensory data processing \cite{clark2024experience}. Therapeutically relevant schemas are generally more abstract predictions about the self, relationships, or whether the world is generally safe/predictable or not.}
  %defined_in={sec:trauma},
  %stem={schema}
}

\newglossaryentry{reconsolidation-exhaustion}{
  name={Reconsolidation Exhaustion},
  description={Sufficient amounts of reconsolidation cause intense mental and physical exhaustion that can feel different from regular exhaustion. It's often called a therapy hangover or EMDR hangover (in EMDR). As far as we can tell, the phenomenon hasn't been formally studied. Common knowledge suggests it dissipates within a few hours to a couple of days. In our experience, people are capable of 1--2\,hours of reconsolidation a day before exhaustion becomes so intense that the process is no longer possible.}
  %defined_in={session},
  %stem={emotionally exhausted, reconsolidation exhaustion, therapy hangover, emotional exhaustion}
}

\newglossaryentry{reconsolidation}{
  name={Reconsolidation},
  description={When a schema/memory first forms, it is \enquote{consolidated} \cite{eckerUnlocking}. When prediction error on that schema becomes large enough, the schema enters a mode where it is changeable. Maintaining that prediction error then updates the schema to reflect the new information. At the end of this process the schema \enquote{re-consolidates,} and becomes unchangeable again. We use \enquote{reconsolidate} to denote this whole process of activation, updating, and reconsolidation.}
  %defined_in={sec:mechanism},
  %stem={reconsolidat}
}

\newglossaryentry{defense-cascade}{
  name={Defense Cascade},
  description={A series of physiological changes in the sympathetic and parasympathetic nervous systems that prepares the body to respond to imminent threats \cite{kozlowskaDefenseCascade}. Includes arousal, flight-or-fight, freezing, and tonic/collapsed immobility. Different situations and experiences activate different responses.}
  %defined_in={sec:defensecascade},
  %stem={defense cascade}
}

\newglossaryentry{arousal}{
  name={Arousal},
  description={The first step when a potential threat is noticed and assists in further assessing that threat \cite{kozlowskaDefenseCascade}. It is also preparation for more intense defense responses like flight-or-fight, freeze, or immobility. Heart rate, breathing, and muscle tone increase, saliva is no longer produced, and core muscles tighten to stabilize posture.}
  %defined_in={sec:defensecascade},
  %stem={arous}
}

\newglossaryentry{flight-or-fight}{
  name={Flight-or-fight},
  description={Active defense response characterized by high levels of adrenaline and muscle activation, increased heart rate, and decreased pain sensitivity \cite{kozlowskaDefenseCascade}.}
  %defined_in={sec:defensecascade},
  %stem={flight-or-fight}
}

\newglossaryentry{freeze}{
  name={Freeze},
  description={A flight-or-fight response temporarily put on hold \cite{kozlowskaDefenseCascade}. One remains highly attentive but frozen to avoid the notice of predators who are more likely to notice moving objects.}
  %defined_in={sec:defensecascade},
  %stem={freez}
}

\newglossaryentry{immobility}{
  name={Tonic/Collapsed Immobility},
  description={Inactive defense responses characterized by detachment, emotional and physical numbing, and immobility \cite{kozlowskaDefenseCascade}. Predators are more attracted to moving prey and may lose interest in seemingly dead bodies. May escalate to unconsciousness.}
  %defined_in={sec:defensecascade},
  %stem={immobil}
}

\newglossaryentry{dissociation}{
  name={Dissociation},
  description={\enquote{Involuntary disruption or discontinuity in the normal integration of one or more of the following: identity, sensations, perceptions, affects, thoughts, memories, control over bodily movements, or behaviour}~\cite{icd11}. Immobility and possibly other states caused by endogenous opioids and parasympathetic activation produce dissociation.}
  %defined_in={sec:defensecascade},
  %stem={dissociat}
}

\newglossaryentry{trauma}{
  name={Trauma},
  description={We use two closely related definitions: First, events that lead to over-generalized schemas that impair functioning or emotional health. Second, distressing events or chronic conditions that overwhelm our ability to cope, where our ability to cope depends on our capabilities and resources \cite{laneReconsolidation}.}
  %defined_in={sec:trauma},
  %stem={trauma}
}

\newglossaryentry{connectogen}{
  name={Connectogen},
  description={A term for MDMA's class of drugs, designed to unify the competing terms \textit{entactogen} and \textit{empathogen} \cite{stockerConnectogen}. MDMA facilitates profound connection to self, body, senses, and others. The ego and sense of self are maintained and hallucinations are minimal, unlike with the classic psychedelics or hallucinogens.}
  %defined_in={sec:introduction},
  %stem={Connectogen}
}

\newglossaryentry{psychedelic}{
  name={Psychedelic},
  description={\enquote{Showing the Mind/Soul,} from the Greek \enquote{psyche} and \enquote{deloun} \cite{osmondObit}. Most are tryptamines, phenethylamines, and ergamides \cite{gumpper2024chemistry}. Each compound has different effects, but they generally relax abstract predictions \cite{carhart2019rebus}. We specifically do not categorize MDMA as a psychedelic, but rather a connectogen \cite{stockerConnectogen}}
  %defined_in={na},
  %stem={psychedelic}
}

\newglossaryentry{attachment-theory}{
  name={Attachment Theory/Styles},
  description={Attachment theory posits that emotionally secure attachments formed in the first 18 months of life serve as the foundation for emotional and psychological development throughout one's life \cite{brownAttachmentDisturbances}. Assess using \textcite{attachmentProject}.}
  %defined_in={attachment},
  %stem={secure attachment, attachment theory, attachment disorder, disordered attachment}
}

\newglossaryentry{predictive-processing}{
  name={Predictive Processing},
  description={The prevailing model of brain function \cite{clark2015surfing}. The brain internally models the world (via complex layers of learned predictions) to better plan for the fulfillment of basic needs such as bodily integrity, reproduction, community, etc. Prediction error is a discrepancy between the brain's model of the world and incoming sensory data or between two contradictory model predictions. Minimization of prediction error is the brain's core optimization function, achieved through the construction of more complex and accurate world models.}
  %defined_in={sec:trauma},
  %stem={predicti}
}

\newglossaryentry{mismatch}{
  name={Mismatch},
  description={The conscious contradiction of an active schema via either sensory input or another schema \cite{eckerUnlocking}.}
  %defined_in={sec:mechanism},
  %stem={mismatch}
}

\newglossaryentry{resistance}{
  name={Resistance},
  description={Opposition to reconsolidation or a broader therapeutic process that would actually be healthy for the individual. This is difficult to ascertain, as many therapeutic processes are not actually a good match for many people.}
  %defined_in={na},
  %stem={resistance}
}

\newglossaryentry{bypass}{
  name={Spiritual Bypass},
  description={As \textcite{cashwell2007Bypass} states, spiritual bypass \enquote{occurs when a person attempts to heal psychological wounds at the spiritual level only and avoids the important (albeit often difficult and painful) work at the other levels, including the cognitive, physical, emotional, and interpersonal.} This results in maladaptive patterns and interrupted psychological development.}
  %defined_in={na},
  %stem={spiritual bypass}
}

\newglossaryentry{avoidance}{
  name={Avoidance},
  description={Physical or mental (often unconscious) actions that direct attention away from contradictory information or distressing thoughts and inhibit reconsolidation \cite{clark2015surfing}. Short-term avoidance can be healthy if used to temporarily postpone dealing with a problem until you have more capacity.}
  %defined_in={sec:trauma},
  %stem={avoida}
}

\newglossaryentry{biopsychosocial-model}{
  name={Biopsychosocial Model},
  description={Prevailing model of mental illness as a complex interaction of biology (genes and medical history), psychology (schemas, in our view), and society (one's support system and social models of how to respond to adversity) \cite{engel1977need}.}
  %defined_in={sec:trauma},
  %stem={biopsychosocial}
}

\newglossaryentry{complex-system}{
  name={Complex System},
  description={Systems with many interacting parts that are difficult to model and understand \cite{hayes2020complex}. The overall behavior of the system can't be easily predicted by looking at the individual components.}
  %defined_in={sec:complex},
  %stem={complex system}
}

\newglossaryentry{attractor-state}{
  name={Attractor},
  description={A state in a complex system that the system is drawn to and which it is difficult to get out of \cite{hayes2020complex}. Addiction is a good example. Maladaptive schemas seem to stick around because they combine with some process inhibiting reconsolidation, which together form an attractor state \cite{hayes2020complex,berghSelfEvidencing}. The durability of the state requires both elements; removing either element destabilizes it.}
  %defined_in={sec:complex},
  %stem={attractor}
}

\newglossaryentry{destabilization}{
  name={Destabilization},
  description={Throughout the book we use the term destabilization as a catchall for two phenomena we describe in \ref{sec:complex}. The first is increased fluctuation between states of good and bad mental health that often precedes a stabilizing transition to the good state. The second is a transition to a stable and even worse state of mental health, possibly precipitated by confronting disturbing memories that were previously avoided. Both phenomena are part of the healthy process of therapy when managed well and entered into at an appropriate time in your life.}
  %defined_in={sec:mechanism},
  %stem={destabiliz}
}

\newglossaryentry{afterglow}{
  name={Afterglow},
  description={A days-to-weeks long period after a session involving well-being, positive mood, mindfulness, positive behaviors, and less mental illness \cite{evansAfterglow}. Regular therapy might be more effective in this period. Common knowledge is that afterglow is unreliable. It can also occasionally be strong enough that people feel like they're on a low dose of MDMA.}
  %defined_in={afterglow},
  %stem={afterglow}
}